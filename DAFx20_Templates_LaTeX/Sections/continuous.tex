\section{Continuous}\label{sec:continuous}
The behaviour of the air in an acoustic tube can be approximated using a 1-dimensional model. 

Consider a tube of \SWcomment[time-varying] length $L$ \SWcomment[$=L(t)$] (in m) defined over spatial domain $x\in [0, L]$ and time $t\geq 0$. A system of first-order PDEs can then be written as
\begin{subequations}\label{eq:firstOrderSystem}
    \begin{align}
        \frac{S}{\rho_0 c^2}\partial_t p &= -\partial_x(Sv)\label{eq:contPressure}\\
        \rho_0\partial_tv &= -\partial_xp\label{eq:discVelocity}
    \end{align}
\end{subequations}
with pressure $p = p(x,t)$ (in N/m$^2$), particle velocity $v = v(x,t)$ (in m/s) and cross-sectional area $S(x)$ (in m$^2$). Furthermore, $\rho_0$ is the density of air (in kg/m$^3$) and $c$ is the speed of sound in air (in m/s).

\subsection{Coupling to a Lip Reed}
To excite the system, a lip reed can be modelled as a simple oscillating mass according to
\begin{equation}
    M_\text{r}\frac{d^2y}{dt^2} = -K y - M_\text{r} \sigma_\text{r} \frac{dy}{dt} + S_\text{r}\Delta p,
\end{equation}
with displacement from the equilibrium $y = y(t)$, lip mass $M_\text{r}$ (in kg), externally supplied (angular) frequency of oscillation $w_0$ \SWcomment[$= w_0(t)$] $= \sqrt{K/M_\text{r}}$ (in rad/s), stiffness $K$ \SWcomment[$= K(t)$] (in N/m), effective surface area $S_\text{r}$ (in m$^2$) and 
\begin{equation}
    \Delta p = P_\text{m} - p(0,t)
\end{equation}
is the difference between the pressure in the mouth $P_\text{m}$ and the pressure in the mouth piece $p(0, t)$ (all in Pa). See Figure \ref{fig:lipSystem} for a schematic of the lip reed model. 
This pressure difference causes a volume flow velocity following the Bernoulli equation
\begin{equation}
    U_\text{B} = w_\text{r}[y + H_0]_+\text{sgn}(\Delta p) \sqrt{\frac{2|\Delta p|}{\rho_0}},
\end{equation}
(in m/s) with effective lip-reed width $w_\text{r}$ (m), static equilibrium separation $H_0$ (in m) and $[\cdot]_+ = 0.5 (\cdot + |\cdot|)$ describes the ``positive part of''. Notice that when $y + H_0 \leq 0$, the lips are closed and the volume velocity $U_\text{B}$ is 0. Another volume flow is generated by the lip reed itself according to
\begin{equation}
    U_\text{r} = S_\text{r} \frac{dy}{dt}
\end{equation}
(in m/s).
Assuming that the volume flow velocity is conserved the total air volume entering the system is defined as
\begin{equation}
    S(0)v(0,t) = U_\text{B}(t) + U_\text{r}(t).
\end{equation}


\input{sections/lipSystem.tex}

\subsection{Radiation}
The radiation model used is the one for the unflanged cylindrical pipe proposed by Levine and Schwinger in \cite{Levine1948} and discretised by Silva \emph{et al.} in \cite{Silva2009}. As it is not important for the contribution of this work it will not be detailed here in full. The reader is instead referred to \cite{Harrison2018} for a comprehensive explanation. 