\begin{table}[t]
    \small
    \begin{center}
    \caption{\it Average CPU usage (in \%) for different graphics settings and various interactions with the application. \label{tab:CPU}}
    \begin{tabular}{|l|c|c|}
        \hline
        Tube length & Graphics (\%) & No graphics (\%)\\\hline
        Short ($L^n = 2.593$ m) & 12.1 & 4.3\\
        Long ($L^n = 3.653$ m) & 14.4 & 5.2 
        \\
        Rapidly changing & 17.7 & 10.1\\\hline    \end{tabular}
    \end{center}
\end{table}

\section{Results and Discussion}\label{sec:resDisc}
The real-time implementation has been tested on a MacBook Pro with a 2.2 GHz Intel i7 processor and was informally evaluated by the authors. The speed of the algorithm was tested with and without the graphics-thread and using three different styles of interaction: static excitation at the shortest and longest length, and rapidly (and continuously) changing $L$ and $\omega_r$ between their minimum and maximum values given in Table \ref{tab:parameters}. The pressure was kept at $P_\text{m} = 3000$ Pa at all times. The results are shown in Table \ref{tab:CPU}. Differences in CPU usage between a short and long tube length are due to the calculation of more grid points in the long case. The recalculation of the geometry maximally once every 20 samples in the rapidly moving case explains the increase in CPU usage there. These results show that the implementation can easily be used as an audio plugin, with or without graphics.

Although the audio output of the simulation exhibits brass-like qualities, it requires some further refinements to be considered as a complete trombone. Possible extensions to improve the realism of the simulation sound could be to add viscothermal losses \cite{Harrison2016} or nonlinear effects \cite{msallam1997physical}.

The method used to implement the dynamic grid does not introduce perceivable audible artefacts, even when $L^n$ is increased and decreased very rapidly. Despite the limit placed on the speed of change of $L^n$ in \eqref{eq:Nmaxdiff} the control of the application does not exhibit a noticeable delay and changes in $L^n$ feel immediate.

The main difference between the method in \cite{Willemsen2021} and the version used here, is that the method is applied to a system of first-order equations rather than the second-order 1D wave equation. Because the connection between the inner boundaries is only applied to one of the two systems, a drift occurs in $w$ as it is left without boundary conditions. Although this drift does not have an effect on the output sound, as discussed in Section \ref{sec:drift}, too high or low values might cause rounding errors in the simulation. As it is expected that this only happens at extremely high or low values after a long simulation length, the drift is not considered an issue at this point. 

% \SWcomment[more for your info, don't think I want to include this:]
% To combat the drift, experiments have been done involving different ways of connecting the left and right tube. One involved alternating between applying the connection to the pressures and the velocity. Here, rather than adding points to the left and right system in alternating fashion, points were added to pressures $p$ and $q$ and velocities $v$ and $w$ in an alternating fashion. Another experiment involved a ``staggered'' version of the connection where (fx.) for one system (either left or right), a virtual grid point of the velocity was created from known values according to \eqref{eq:connectionInterpol}, rather than both from pressures. This, however, showed unstable behaviour. No conclusory statements can be made about these experiments at this point. \SWcomment[$\leftarrow$ which is exactly why I don't want to include this section]



As the geometry varies it matters a lot where points are added and removed as this might influence the way that the method is implemented. \SWcomment[speculative section coming up] The middle of the slide crook was chosen, both because it would be reasonable for the air on the tube to ``go away from'' or ``go towards'' that point as the slide is extended or contracted, and because the geometry does not vary there. Experiments with adding / removing grid points where the geometry varies have been left for future work. \SWcomment[even more speculative.. $\rightarrow$] It could be argued that it makes more sense to add points at the ends of the inner slides as ``tube material'' is also added there. This would mean that the system should be split in three parts: ``inner slide", ``outer slide" and ``rest", and would complicate things even more.

