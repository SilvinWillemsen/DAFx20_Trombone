\section{Results and Discussion}\label{sec:resDisc}
The real-time implementation has been informally evaluated by the authors. 

No audible artefacts are observed, even when $L^n$ is  

The evaluation concluded that although the sound has brass-like qualities, it does not sound fully realistic. Possible extensions to make the simulation sound more realistic are to add viscothermal losses \cite{Harrison2016} or nonlinear effects \cite{msallam1997physical}.

The limit placed on the speed of change of $L^n$ in \eqref{eq:Nmaxdiff} does not affect the feeling of real-time change when controlling the application.

The main difference with \cite{Willemsen2021}, is that the method is applied to a system of first-order equations rather than the second-order 1D wave equation. Because the connection between the inner boundaries is only applied to one of the two systems, a drift occurs in $w$ as it is left without boundary conditions. Although, as discussed in Section \ref{sec:drift}, this drift does not have an effect on the output sound, too high or low values might cause numerical inaccuracies. As it is expected that this only happens at extremely high or low values after a long simulation length, the drift is not considered an issue at this point. 

\SWcomment[more for your info, don't think I want to include this:]
To combat the drift, experiments have been done involving different ways of connecting the left and right tube. One involved alternating between applying the connection to the pressures and the velocity. Here, rather than adding points to the left and right system in alternating fashion, points were added to pressures $p$ and $q$ and velocities $v$ and $w$ respectively. Another experiment involved a ``staggered'' version of the connection where (fx.) for one system (either left or right), a virtual grid point of the velocity was created from known values according to \eqref{eq:connectionInterpol}, rather than both from pressures. This, however, showed unstable behaviour. No conclusory statements can be made about these experiments at this point. \SWcomment[$\leftarrow$ which is exactly why I don't want to include this section]



As the geometry varies it matters a lot where points are added and removed as this might influence the way that the method is implemented. \SWcomment[speculative section coming up] The middle of the slide crook was chosen, both because it would be reasonable for the air on the tube to ``go away from'' or ``go towards'' that point as the slide is extended or contracted, and because the geometry does not vary there. Experiments with adding / removing grid points where the geometry varies have been left for future work. \SWcomment[even more speculative.. $\rightarrow$] It could be argued that it makes more sense to add points at the ends of the inner slides as ``tube material'' is also added there. This would mean that the system should be split in three parts: ``inner slide", ``outer slide" and ``rest", and would complicate things even more.

