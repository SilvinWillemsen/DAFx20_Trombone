\section{Continuous System}\label{sec:continuous}
Wave propagation in an acoustic tube can be approximated using a 1-dimensional model. Consider a tube of \SWcomment[time-varying] length $L$ \SWcomment[$=L(t)$] (in m) defined over spatial domain $x\in [0, L]$ and time $t\geq 0$. Using operators $\partial_t$ and $\partial_x$ denoting a first-order derivative with respect to time $t$ and space $x$, respectively, a system of first-order PDEs describing the wave propagation in an acoustic tube can then be written as
\begin{subequations}\label{eq:firstOrderSystem}
    \begin{align}
        \frac{S}{\rho_0 c^2}\partial_t p &= -\partial_x(Sv)\label{eq:contPressure}\\
        \rho_0\partial_tv &= -\partial_xp\label{eq:contVelocity}
    \end{align}
\end{subequations}
with pressure $p = p(x,t)$ (in N/m$^2$), particle velocity $v = v(x,t)$ (in m/s) and (circular) cross-sectional area $S(x)$ (in m$^2$). Furthermore, $\rho_0$ is the density of air (in kg/m$^3$) and $c$ is the speed of sound in air (in m/s).

Boundary conditions can then be imposed at the ends of domain, $x=0, L$. We assume the left boundary (at the mouthpiece) to be closed and the right (at the bell) to be open according to  
\begin{subequations}\label{eq:contBound}
    \begin{align}
        S(0,t)v(0,t) &= 0, \quad \text{(Neumann, closed)}\label{eq:contNeumann}\\
        p(L,t) &= 0. \quad \text{(Dirichlet, open)}\label{eq:contDirichlet}
    \end{align}
\end{subequations}
In the following, these (lossless) boundary conditions will be modified to be coupled to a lip reed and radiating respectively.

\subsection{Coupling to a Lip Reed}
To excite the system, a lip reed can be modelled as a simple oscillating mass according to
\begin{equation}\label{eq:lipReedCont}
    M_\text{r}\frac{d^2y}{dt^2} = -M_\text{r}\omega_0^2 y - M_\text{r} \sigma_\text{r} \frac{dy}{dt} + S_\text{r}\Delta p,
\end{equation}
with displacement from the equilibrium $y = y(t)$, lip mass $M_\text{r}$ (in kg), externally supplied (angular) frequency of oscillation $w_0$ \SWcomment[$= w_0(t)$] $= \sqrt{K/M_\text{r}}$ (in rad/s), stiffness $K$ \SWcomment[$= K(t)$] (in N/m), effective surface area $S_\text{r}$ (in m$^2$) and 
\begin{equation}
    \Delta p = P_\text{m} - p(0,t)
\end{equation}
is the difference between the pressure in the mouth $P_\text{m}$ and the pressure in the mouth piece $p(0, t)$ (all in Pa). See Figure \ref{fig:lipSystem} for a schematic of the lip reed model. 
This pressure difference causes a volume flow velocity following the Bernoulli equation
\begin{equation}
    U_\text{B} = w_\text{r}[y + H_0]_+\text{sgn}(\Delta p) \sqrt{\frac{2|\Delta p|}{\rho_0}},
\end{equation}
(in m/s) with effective lip-reed width $w_\text{r}$ (m), static equilibrium separation $H_0$ (in m) and $[\cdot]_+ = 0.5 (\cdot + |\cdot|)$ describes the ``positive part of''. Notice that when $y + H_0 \leq 0$, the lips are closed and the volume velocity $U_\text{B}$ is 0. Another volume flow is generated by the lip reed itself according to
\begin{equation}
    U_\text{r} = S_\text{r} \frac{dy}{dt}
\end{equation}
(in m/s).
Assuming that the volume flow velocity is conserved, the total air volume entering the system is defined as
\begin{equation}\label{eq:lipBoundary}
    S(0)v(0,t) = U_\text{B}(t) + U_\text{r}(t).
\end{equation}
The lip reed can then be coupled to the tube by modifying boundary condition \eqref{eq:contNeumann} to \eqref{eq:lipBoundary}.

\begin{figure}[ht]
    \centering
    \begin{tikzpicture}
    
    \def\radius{6}; % Radius of the string (>2!)
    \pgfmathsetmacro{\reps}{3}; % How may back-and-forths in the drawing of the springs
    \def\bowSpacing{0.2};
    \def\drawingSpacing{1.5}
    \def\bowWidth{5};
    
    \def\woodWidth{1}; %>0.3
    \def\massWidth{2};
    \def\bridgeHeight{3};
    \def\bridgeWidth{4};
    \def\cornerRadius{0.15};
    \def\stringWidth{0.2};
    \pgfmathsetmacro{\tinyRadius}{\stringWidth*0.1};
    \pgfmathsetmacro{\stringWidthMinTinyRad}{((\stringWidth-(2*\tinyRadius)))*0.5};
    
    \def\axisLineWidth{0.07};

    % draw airflow
    
    %draw airflow
    \def\rightAirFlow{0}; % have the right airflow bulge (1) or not (0)
    \foreach \idx in {1,...,5}
    {
        \pgfmathsetmacro{\scaleLeft}{0.5 - 0.1 * \idx};
        \ifnum\rightAirFlow=1
            \pgfmathsetmacro{\scaleRight}{0.5 - 0.1 * \idx};
        \else
            \pgfmathsetmacro{\scaleRight}{0};
        \fi
        \begin{scope}[decoration={
            markings,
            mark=between positions 0.15 and 0.85 step 0.35
         with {\arrow{>}}}
            ]
        % \node at (0, \idx) {\scale};
         \draw [gray!40, 
         xshift=-2.5cm, 
         yshift= -\idx * 0.3cm, 
         dotted, 
         line width=0.3mm, postaction={decorate}] plot [smooth, tension = 0.5] coordinates { (0,1*\scaleLeft) (1,0.75*\scaleLeft) (2,0) (4, 0) (5, 0.75*\scaleRight) (6,1*\scaleRight)} ;
         \end{scope}
    }
    % \def\scale{0.5};
    % \draw [gray, xshift=-2.5cm, yshift=-0cm] plot [smooth, tension = 0.5] coordinates { (0,1*\scale) (1,0.75*\scale) (2,0) (4, 0) (5, 0.75*\scale) (6,1*\scale)};
    % \def\scale{0.25};
    % \draw [gray, xshift=-2.5cm, yshift=-1.2cm] plot [smooth, tension = 0.5] coordinates { (0,1*\scale) (1,0.75*\scale) (2,0) (4, 0) (5, 0.75*\scale) (6,1*\scale)};
    % \def\scale{0};
    % \draw [gray, xshift=-2.5cm, yshift=-1.5cm] plot [smooth, tension = 0.5] coordinates { (0,1*\scale) (1,0.75*\scale) (2,0) (4, 0) (5, 0.75*\scale) (6,1*\scale)};
    
    \node (r0) at ( 1.0,  -0.5 ) {}; % root
    \node (s0) at ( 1.0, 0.1 ) {}; % extreme
    \node (s1) at ( 1.6, 0.1 ) {}; % extreme

    % DRAW TREE
    \fill[fill=white] (r0.center)--(s0.center)--(s1.center);
    
    % \draw plot [smooth] coordinates {(-3, -3) (-2, 2) (-1, -3};
    \node at (0,0) [rectangle,draw, fill = white, minimum height=1cm,minimum width= \massWidth cm] (Mr) {$M_\text{r}$};
    %top
    % \draw[-] (-3, 2) -- (3, 2) node[below, midway] (top) {};
    % \draw[-] (-2, 3) -- (4, 3) node[below, midway] (top) {};
    % \draw[-] (-3, 2) -- (-2, 3) node[below, midway] (top) {};
    % \draw[-] (4, 3) -- (3, 2) node[below, midway] (top) {};
    \draw[-] (-2.5, 2.5) -- (3.5, 2.5) node[below, midway] (top) {};
    %bottom
    \draw[-] (-3, -2) -- (3, -2) node[below, midway] (bottom) {};
    \draw[-] (-2, -1) -- (4, -1) node[below, midway] (bottom) {};
    \draw[-] (-3, -2) -- (-2, -1) node[below, midway] (top) {};
    \draw[-] (4, -1) -- (3, -2) node[below, midway] (top) {};

    % draw mass
    
    \draw[-] (-1, 0.5) -- (0, 1.5) node[] (left) {};

    \draw[-] (1, 0.5) -- (2, 1.5) node[] (topRight) {};

    \draw[-] (0, 1.5) -- (2, 1.5) node[] (top) {};

    \draw[-] (2, 1.5) -- (2, 0.5) node[] (right) {};

    \draw[-] (0.5 * \massWidth, -0.5) -- (2, 0.5) node[] (bottomRight) {};

    \node[rotate = 0] at (0.5, 1) (Sr) {$S_\text{r}$};

    % eta line
    \draw[-, color = black] (0.5*\massWidth+2*\axisLineWidth, -0.5) -- (0.5*\massWidth+2*\axisLineWidth, -1.95)
    node[anchor = west] at (0.5*\massWidth +0.07, -1.33) (eta) {$-\eta$};

    \draw[-] (0.5 * \massWidth + \axisLineWidth, -0.5) -- (0.5 * \massWidth + 3*\axisLineWidth, -0.5) {}; 
    
    \draw[-] (0.5 * \massWidth +\axisLineWidth, -1.95) -- (0.5 * \massWidth +3*\axisLineWidth, -1.95); 
    \def\xOffset{0.3};
    % draw spring
    \filldraw[black] (-0.5 + \xOffset, 2.5) circle (1pt) node[anchor=center](topSpring){};    
    \draw[-] (-0.5 + \xOffset, 2.5) -- (-0.5 + \xOffset, 2.3);
    \draw[-] (-0.5 + \xOffset, 2.3) -- (-0.75 + \xOffset, 2.2);
    %switched these around because of the color
    \draw[-] (-0.25 + \xOffset, 2.02) -- (-0.75 + \xOffset, 1.84);
    \draw[-] (-0.75 + \xOffset, 2.2) -- (-0.25 + \xOffset, 2.02);
    \draw[-] (-0.75 + \xOffset, 1.84) -- (-0.25 + \xOffset, 1.66);
    \draw[-] (-0.25 + \xOffset, 1.66) -- (-0.75 + \xOffset, 1.52);
    \draw[-] (-0.75 + \xOffset, 1.52) -- (-0.25 + \xOffset, 1.3);
    \draw[-] (-0.25 + \xOffset, 1.3) -- (-0.5 + \xOffset, 1.2);
    \draw[-] (-0.5 + \xOffset, 1.2) -- (-0.5 + \xOffset, 1.0);
    \filldraw[black] (-0.5 + \xOffset, 1.0) circle (1pt) node[anchor=center](bottomSpring){};    
    \node at (-1 + \xOffset, 1.75) (K) {$K$};
    
    \def\dashpotHeight{-0.25}
    % draw dashpot
    \filldraw[black] (1.5 - \xOffset, 2.5) circle (1pt) node[anchor=center](topDashPot){};
    \draw[-] (1.5 - \xOffset, 2.5) -- (1.5 - \xOffset, 1.4 - \dashpotHeight);

    \draw[-] (1.3 - \xOffset, 1.4 - \dashpotHeight) -- (1.7 - \xOffset, 1.4 - \dashpotHeight);

    \draw[-] (1.25 - \xOffset, 1.7 - \dashpotHeight) -- (1.25 - \xOffset, 1.35 - \dashpotHeight);
    \draw[-] (1.75 - \xOffset, 1.35 - \dashpotHeight) -- (1.75 - \xOffset, 1.7 - \dashpotHeight);

    \draw[-] (1.25 - \xOffset, 1.35 - \dashpotHeight) -- (1.75 - \xOffset, 1.35 - \dashpotHeight);
    \draw[-] (1.5 - \xOffset, 1.35 - \dashpotHeight) -- (1.5 - \xOffset, 1.0);

    \filldraw[black] (1.5 - \xOffset, 1.0) circle (1pt) node[anchor=center](bottomDashpot){};    
    \node at (1.0 - \xOffset, 1.75) (sigma) {$\sigma_\text{r}$};
    
    
    % pressure labels
    \def\pressOffset{0.3}
    \def\backgroundOpacity{0.4}
    \node[fill = white, fill opacity=\backgroundOpacity, text opacity = 1] at (-2, -0.6 - \pressOffset) (Pm) {$P_\text{m}$};
    \node[fill = white, fill opacity=\backgroundOpacity, text opacity = 1] at (0.5, -0.8 - \pressOffset) (deltaP) {$\Delta p$};
    \node[fill = white, fill opacity=\backgroundOpacity, text opacity = 1] at (3.2, -0.6 - \pressOffset) (p) {$p(0,t)$};

    
    % y and H0
    \draw[dashed, color = gray] (3.65, 0) -- (1.5, 0);
    \node at (4, 1.5) {$y$};
    
    \draw[->] (3.75, -1.5) -- (3.75, 1.5);
    \node at (4, 0) {$0$};
    \draw (3.75 - \axisLineWidth, 0) -- (3.75 + \axisLineWidth, 0) {};


    \draw (3.75 - \axisLineWidth, -1.5) -- (3.75 + \axisLineWidth, -1.5) {};
    \node at (4.2, -1.5) {$-H_0$};
    
    % width
    \draw[black!70] (-1, 0.6) -- (1, 0.6) {};
    \draw[black!70] (-1, 0.55) -- (-1, 0.65) {};
    \draw[black!70] (1, 0.55) -- (1, 0.65) {};
    \node at (0, 0.75) (w) {$w$};
% \begin{scope}[very thick,decoration={
%     markings,
%     mark=at position 0.5 with {\arrow{>}}}
%     ] 
%     \draw[postaction={decorate}] (-4,0)--(4,0);
% \end{scope}
    
    \end{tikzpicture}
    \caption{Lipsystem with the equilibrium at 0 and the distance from the lower lip $H_0$. The various symbols relate to those used in Eq. \eqref{eq:lipReedCont}.}
    \label{fig:lipSystem}
\end{figure}

\subsection{Radiation}
As the bell-end of brass instruments is in a way ``coupled" to the air, the tube loses energy at the bell, or right boundary. These losses can be modelled using a radiation model and lossless condition \eqref{eq:contDirichlet} can be modified to be radiating instead. The radiation model used is the one for the unflanged cylindrical pipe proposed by Levine and Schwinger in \cite{Levine1948} and discretised by Silva \emph{et al.} in \cite{Silva2009}. As this model is not important for the contribution of this work it will not be detailed here in full. The interested reader is instead referred to \cite{Harrison2018} for a comprehensive explanation. 