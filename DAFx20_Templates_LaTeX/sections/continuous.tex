\section{Continuous System}\label{sec:continuous}
Wave propagation in an acoustic tube can be approximated using a 1-dimensional model. Consider a tube of \SWcomment[time-varying] length $L$ \SWcomment[$=L(t)$] (in m) defined over spatial domain $x\in [0, L]$ and time $t\geq 0$. Using operators $\partial_t$ and $\partial_x$ denoting a first-order derivative with respect to time $t$ and space $x$, respectively, a system of first-order PDEs describing the wave propagation in an acoustic tube can then be written as
\begin{subequations}\label{eq:firstOrderSystem}
    \begin{align}
        \frac{S}{\rho_0 c^2}\partial_t p &= -\partial_x(Sv)\label{eq:contPressure}\\
        \rho_0\partial_tv &= -\partial_xp\label{eq:contVelocity}
    \end{align}
\end{subequations}
with pressure $p = p(x,t)$ (in N/m$^2$), particle velocity $v = v(x,t)$ (in m/s) and (circular) cross-sectional area $S(x)$ (in m$^2$). Furthermore, $\rho_0$ is the density of air (in kg/m$^3$) and $c$ is the speed of sound in air (in m/s).

Boundary conditions can then be imposed at the ends of domain, $x=0, L$. We assume the left boundary (at the mouthpiece) to be closed and the right (at the bell) to be open according to  
\begin{subequations}\label{eq:contBound}
    \begin{align}
        S(0,t)v(0,t) &= 0, \quad \text{(Neumann, closed)}\label{eq:contNeumann}\\
        p(L,t) &= 0. \quad \text{(Dirichlet, open)}\label{eq:contDirichlet}
    \end{align}
\end{subequations}
In the following, these (lossless) boundary conditions will be modified to be coupled to a lip reed and radiating respectively.

\subsection{Coupling to a Lip Reed}
To excite the system, a lip reed can be modelled as a simple mass-spring-damper system. In the following, $y$ can be seen as the moving the upper lip where the lower lip is left static and rigid. See Figure \ref{fig:lipSystem} for a full schematic of the lip reed model. Using dots to indicate time-derivatives the lip reed is modelled as
\begin{equation}\label{eq:lipReedCont}
    M_\text{r}\ddot y = -M_\text{r}\omega_0^2 y - M_\text{r} \sigma_\text{r} \dot y +\psi(\dot \psi/
\dot \eta)+ S_\text{r}\Delta p,
\end{equation}
with displacement from the equilibrium $y = y(t)$, lip mass $M_\text{r}$ (in kg), externally supplied (angular) frequency of oscillation $w_0$ \SWcomment[$= w_0(t)$] $= \sqrt{K/M_\text{r}}$ (in rad/s) and stiffness $K$ \SWcomment[$= K(t)$] (in N/m).

We then introduce a nonlinear collision between the lips using potential
\begin{equation}
    \psi = \left(\frac{2K_\text{c}}{\alpha_\text{c}+1}[-\eta]_+^{\alpha_\text{c}+1}
    \right)^{1/2}
\end{equation}
\begin{equation*}
    K_\text{c}>0, \quad \alpha_\text{c}\geq 1, \quad \eta\triangleq y+H_0
\end{equation*}
with collision stiffness $K_\text{c}$ (in N/m if $\alpha_\text{c} = 1$) dimensionless nonlinear collision coefficient $\alpha_\text{c}$, distance between the lips $\eta = \eta(t)$ (in m), $[\eta]_+ = 0.5 (\eta + |\eta|)$ describing the ``positive part of $\eta$'',  and static equilibrium separation $H_0$ (in m).

Finally, $S_\text{r}$ (in m$^2$) is the effective surface area and 
\begin{equation}
    \Delta p = P_\text{m} - p(0,t)
\end{equation}
is the difference between the pressure in the mouth $P_\text{m}$ and the pressure in the mouth piece $p(0, t)$ (all in Pa). This pressure difference causes a volume flow velocity following the Bernoulli equation
\begin{equation}
    U_\text{B} = w_\text{r}[\eta]_+\text{sgn}(\Delta p) \sqrt{\frac{2|\Delta p|}{\rho_0}},
\end{equation}
(in m/s) with effective lip-reed width $w_\text{r}$ (m). Notice that when $\eta \leq 0$, the lips are closed and the volume velocity $U_\text{B}$ is 0. Another volume flow is generated by the lip reed itself according to
\begin{equation}
    U_\text{r} = S_\text{r} \frac{dy}{dt}
\end{equation}
(in m/s).
Assuming that the volume flow velocity is conserved, the total air volume entering the system is defined as
\begin{equation}\label{eq:lipBoundary}
    S(0)v(0,t) = U_\text{B}(t) + U_\text{r}(t).
\end{equation}
The lip reed can then be coupled to the tube by modifying boundary condition \eqref{eq:contNeumann} to \eqref{eq:lipBoundary}.

\input{sections/lipSystem.tex}

\subsection{Radiation}
As the bell-end of brass instruments is in a way ``coupled" to the air, the tube loses energy at the bell, or right boundary. These losses can be modelled using a radiation model and lossless condition \eqref{eq:contDirichlet} can be modified to be radiating instead. The radiation model used is the one for the unflanged cylindrical pipe proposed by Levine and Schwinger in \cite{Levine1948} and discretised by Silva \emph{et al.} in \cite{Silva2009}. As this model is not important for the contribution of this work it will not be detailed here in full. The interested reader is instead referred to \cite{Harrison2018} for a comprehensive explanation. 