\section{Discretisation}\label{sec:discrete}
The continuous system described in the previous section is discretised using FDTD methods, through an approximation over a grid in space and time. Before presenting this discretisation, we briefly summarize the operation of FDTD methods.

\subsection{Numerical Methods}\label{sec:numMeth}
Consider a 1D system of length $L$ described by state variable $u = u(x,t)$ with spatial domain $x\in [0, L]$ and time $t\geq 0$. The spatial domain can be disctretised according to $x=lh$ with spatial index $l \in [0, \hdots, N]$, number of intervals between the grid points $N$, grid spacing $h$ (in m) and time as $t=nk$ with temporal index $n \in \mathbb{Z}^{0+}$ and time step $k$ (in s). The grid function $u_l^n$ represents an approximation to $u(x,t)$ at $x=lh$ and $t=nk$. 

Shift operators can then be applied to grid function $u_l^n$. Temporal and spatial shift operators are
\begin{equation}
    \begin{aligned}
        e_{t+}u_l^n &= u_l^{n+1}, \quad e_{t-}u_l^n = u_l^{n-1},\\
        e_{x+}u_l^n &= u_{l+1}^n\;, \quad \!e_{x-}u_l^n = u_{l-1}^n,
    \end{aligned}
\end{equation}
from which more complex operators can be derived.
First-order derivatives can be approximated using forward, backward and centred difference operators in time
\begin{equation}\label{eq:discTimeOperators}
    \delta_{t+} = \frac{e_{t+} - 1}{k},\ \delta_{t-} = \frac{1 - e_{t-}}{k},\ \delta_{t\cdot} = \frac{e_{t+}-e_{t-}}{2k},
\end{equation}
(all approximating $\partial_t$) and space
\begin{equation}\label{eq:discSpaceOperators}
    \delta_{x+} = \frac{e_{x+} - 1}{h},\ \delta_{x-} = \frac{1 - e_{x-}}{h},\ \delta_{x\cdot} = \frac{e_{x+}-e_{x-}}{2h},
\end{equation} 
(all approximating $\partial_x$) where $1$ is the identity operator.

Furthermore, forward, backward and centred averaging operators can be defined in time
\begin{equation}
    \mu_{t+} = \frac{e_{t+} + 1}{2}, \ \mu_{t-} = \frac{1 + e_{t-}}{2}, \ \mu_{t\cdot} = \frac{e_{t+}+e_{t-}}{2},
\end{equation}
and space
\begin{equation}
    \mu_{x+} = \frac{e_{x+} + 1}{2}, \ \mu_{x-} = \frac{1 + e_{x-}}{2}, \ \mu_{x\cdot} = \frac{e_{x+}+e_{x-}}{2}.
\end{equation}
Here, forward and backward averaging operators are extremely useful in the context of first-order systems as used in this paper. When applied to a grid function, the result may be interpreted as its value shifted by half a temporal or spatial step: 
\begin{align}
    \mu_{t+}u_l^n &= u_l^{n+1/2}, \quad \mu_{t-}u_l^n = u_l^{n-1/2},\\
    \mu_{x+}u_l^n &= u_{l+1/2}^n, \quad \mu_{x-}u_l^n = u_{l-1/2}^n,
\end{align}
effectively placing the grid function on an \textit{interleaved grid} which will be further elaborated on in the following. \SBcomment[OK, the interleaved averaging expressions above aren't right...do you need this? Can't you just say that the operators $\mu_{t\pm}$ apply equally to interleaved grid functions and leave it at that? At the moment, the definition above seems like it is self-contradictory...] \SWcomment[I think what I was getting at is that you can put a grid function on the interleaved grid using these averaging operators. After some thinking though, this is more an approximation rather than an equality (just like $S_{l+1/2}$ and $\bar S_l$). The statement "the result may be interpreted as its value shifted by half a step" is therefore also wrong no?]
 
An approximation $\delta_{tt}$ to a second time derivative may be defined as
\begin{equation}
    \delta_{tt} = \delta_{t+}\delta_{t-} = \frac{1}{k^2}\left(e_{t+}-2+e_{t-}\right).
\end{equation}

\subsection{Discrete Tube}\label{sec:discSyst}
As a first step, the domain $x\in [0, L]$ can be subdivided into $N$ equal segments of length $h$ (the grid spacing), so that
\begin{equation}\label{eq:numberOfIntervals}
    N=L/h.
\end{equation}
Interleaved grid functions approximating $p$ and $v$ may then be defined. $p_l^n$ $l=0,\hdots,N$ \SWcomment[Just a question about notation here.. Is there a difference between $l = {[}0, \hdots, N{]}$, $l = 0, \hdots, N$ and $l\in{[}0, \hdots, N{]}$ and when to use which?] approximates $p(x,t)$ at coordinates $x=lh$, $t=nk$ and $v_{l+1/2}^{n+1/2}$, $l=0\hdots,N-1$ approximates $v(x,t)$ at coordinates $x=(l+1/2)h$, $t=(n+1/2)k$. In addition, a discrete cross-sectional area $S_l\approx S(x=lh)$, $l=0,\hdots,N$ is assumed known.

System \eqref{eq:firstOrderSystem} can then be discretised as
\begin{subequations}\label{eq:FDS}
    \begin{align}
        \frac{\bar S_l}{\rho_0 c^2}\delta_{t+}p_l^n &= -\delta_{x-}(S_{l+1/2}v_{l+1/2}^{n+1/2}),\label{eq:discPressure}\\
        \rho_0 \delta_{t-}v_{l+1/2}^{n+1/2}&=-\delta_{x+}p_l^n,\label{eq:discVelocity}
    \end{align}
\end{subequations}
where $S_{l+1/2} = \mu_{x+}S_l$ and $\bar S_l = \mu_{x-}S_{l+1/2}$ are approximations to the continuous cross-sectional area $S(x)$. The values for $\bar S_l$ at the boundaries, i.e., $\bar S_0$ and $\bar S_N$ are set equal to $S(0)$ and $S(L)$.

Expanding the operators, we obtain the following recursion
\begin{subequations}\label{eq:updateNormal}
    \begin{align}
        p_l^{n+1} &= p_l^n - \frac{\rho_0 c \lambda}{\bar{S}_l}(S_{l+1/2}v_{l+1/2}^{n+1/2}-S_{l-1/2}v_{l-1/2}^{n+1/2}),\label{eq:pressureUpdate}\\
        v_{l+1/2}^{n+1/2} &= v_{l+1/2}^{n-1/2}-\frac{\lambda}{\rho_0 c}(p_{l+1}^n - p_l^n),\label{eq:velocityUpdate}
    \end{align}
\end{subequations}
where $\lambda = ck/h$ is referred to as the Courant number and
\begin{equation}\label{eq:CFL}
    \lambda \leq 1
\end{equation}
in order for the scheme to be stable \cite{}. \SBcomment[OK, right here would be the place to introduce the truncation of $N$, with reference to the CFL condition...]

The update \eqref{eq:velocityUpdate} holds for $l=0,\hdots,N-1$. The update \eqref{eq:pressureUpdate} holds for $l=0,\hdots,N$, and thus, in analogy with the continuous case, two numerical boundary conditions are required in order to update $p_{0}^{n+1}$ and $p_{N}^{n+1}$. These are provided by numerical equivalents of the excitation and radiation condition, given below. 
% Finally, the boundary conditions in \eqref{eq:contBound} can be discretised as
% \begin{subequations}
%     \begin{align}
%         \mu_{x-}\left(S_{1/2}v_{1/2}^{n+1/2}\right) = 0\label{eq:discNeumann}\\
%         p_N^n = 0\label{eq:discDirich}
%     \end{align}    
% \end{subequations}

\subsection{Lip reed}
As the lip reed interacts with the particle velocity of the tube, it is placed on the interleaved temporal grid, but kept on the regular spatial grid, as it interacts with the boundary at $l=0$. Equations \eqref{eq:lipReedCont} - \eqref{eq:lipBoundary} are then discretised as follows:
\def\nphSys{n+1/2}
\begin{subequations}\label{eq:discreteLipSystem}
    \begin{align}
    &\begin{aligned}
        M_\text{r}\delta_{tt}y^{\nphSys} =&-M_\text{r}\omega_0^2\mu_{t\cdot}y^{\nphSys} -M_\text{r}\sigma_\text{r}\delta_{t\cdot}y^{\nphSys}\\
        &+\left(\mu_{t+}\psi^n\right)g^{n+1/2}+S_\text{r}\Delta p^{\nphSys},
    \end{aligned}\label{eq:discReed}\\
    &\Delta p^{\nphSys} = P_\text{m} - \mu_{t+}p_0^n,\label{eq:pDiff}\\
    &\begin{aligned}
        U_\text{B}^{\nphSys} =&\ w_\text{r}[-\eta^{\nphSys}]_+\text{sgn}(\Delta p^{\nphSys})\label{eq:bernoulli}\\
        &
        \cdot\sqrt{2|\Delta p^{\nphSys}|/\rho_0},
    \end{aligned}\\
    &U_\text{r}^{\nphSys}= S_\text{r}\delta_{t\cdot}y^{\nphSys},\label{eq:Ur}\\
    &\mu_{x-}(S_{1/2}v_{1/2}^{\nphSys})= U_\text{B}^{\nphSys} + U_\text{r}^{\nphSys},\label{eq:UbUr}
    \end{align}
\end{subequations}
where $\omega_0 = \omega_0^{n+1/2}$ and $P_\text{m} = P_\text{m}^{n+1/2}$ and
\begin{subnumcases}{ \label{eq:gDef} g^{n+1/2} =}
\begin{aligned}
\kappa&\sqrt{\frac{K_\text{c}(\alpha+1)}{2}}\\
&\cdot(\eta^{n+1/2})^{\frac{\alpha-1}{2}}
\end{aligned} & if $\eta^{n+1/2} \geq 0$ \label{eq:collCorr1}\\
-2 \frac{\psi^n}{\eta^\star-\eta^{n-1/2}} & if $\eta^{n+1/2} < 0$\label{eq:collCorr2}\\
0, & $\begin{aligned} &\text{if } \eta^{n+1/2} < 0\\ &\text{and } \eta^{\star} = \eta^{n-1/2}
\end{aligned}$\label{eq:collCorr3}
\end{subnumcases}
%
where $\kappa = 1$ if $\psi^n \geq 0$, otherwise $\kappa = -1$. It should be noted that condition \eqref{eq:collCorr3} has been added to the definition of $g$ from \cite{Ducceschi2021} to prevent a division by 0 in \eqref{eq:collCorr2}. Finally, $\eta^\star = -y^{\star} - H_0$ where $y^{\star}$ is the value of $y^{n+3/2}$ calculated using system \eqref{eq:discreteLipSystem} (after expansion) without the collision potential. This means that system \eqref{eq:discreteLipSystem} needs to be calculated twice every iteration, once without the collision term and once with. The process of calculating the pressure difference $\Delta p^{n+1/2}$ in \eqref{eq:discreteLipSystem} will not be given here, but the interested reader is referred to \cite[Ch. 5]{Harrison2018} for a derivation.

% Finally, to couple the lip reed to the tube, boundary condition \eqref{eq:discNeumann} can be modified to Eq. \eqref{eq:UbUr}. \SBcomment[Another reference to the Neumann condition to possibly remove here?] 
% \begin{subnumcases}{\label{eq:discreteLipSystem}}
%     M_\text{r}\delta_{tt}y^{\nphSys} = -M_\text{r}\omega_0^2\mu_{t\cdot}y^{\nphSys} -M_\text{r}\sigma_\text{r}\delta_{t\cdot}y^{\nphSys}\label{eq:discReed}\\\qquad \qquad \qquad \ \ \ 
%     +\ S_\text{r}\Delta p^{\nphSys}\nonumber\\
%     \Delta p^{\nphSys} = P_\text{m} - \mu_{t+}p_0^n\label{eq:pDiff}\\
%     U_\text{B}^{\nphSys} = w[y^{\nphSys}+H_0]_+\text{sgn}(\Delta p^{\nphSys})\label{eq:bernoulli}\\\qquad \qquad \ \ \ 
%     \cdot\ \sqrt{2|\Delta p^{\nphSys}|/\rho_0}\nonumber\\
%     U_\text{r}^{\nphSys}= S_\text{r}\delta_{t\cdot}y^{\nphSys}\label{eq:Ur}\\
%     \mu_{x-}(S_{1/2}v_{1/2}^{\nphSys})= U_\text{B}^{\nphSys} + U_\text{r}^{\nphSys}\label{eq:UbUr}
% \end{subnumcases}

% \begin{subnumcases}{\label{eq:discreteLipSystem}}
%     M_\text{r}\delta_{tt}y^{\nphSys} & $= -M_\text{r}\omega_0^2\mu_{t\cdot}y^{\nphSys}-M_\text{r}\sigma_\text{r}\delta_{t\cdot}y^{\nphSys} + S_\text{r}\Delta p^{\nphSys}$\label{eq:discReed}\\
%     \Delta p^{\nphSys} & $= P_\text{m} - \mu_{t+}p_0^n$\label{eq:pDiff}\\
%     U_\text{B}^{\nphSys} & $= w[y^{\nphSys}+H_0]_+\text{sgn}(\Delta p^{\nphSys})\sqrt{\frac{2|\Delta p^{\nphSys}|}{\rho_0}}$\label{eq:bernoulli}\\
%     U_\text{r}^{\nphSys} & $= S_\text{r}\delta_{t\cdot}y^{\nphSys}$\label{eq:Ur}\\
%     \mu_{x-}(S_{1/2}v_{1/2}^{\nphSys})\!\!\!\!\! \!\!\!\!\! &$= U_\text{B}^{\nphSys} + U_\text{r}^{\nphSys}$\label{eq:UbUr}
% \end{subnumcases}