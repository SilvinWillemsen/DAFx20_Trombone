\section{Introduction}\label{sec:introduction}

The trombone is a musical instrument which is interesting from the simulation perspective from different viewpoints.
From the point of view of the excitation, the interaction between the lips and the player has been extensively studied, and simulated mostly using a simple mass-spring damper system \cite{campbell2004brass}.
The sound propagation in the trombone also presents some very interesting nonlinearities, which have been investigated and simulated
\cite{campbell2004brass, msallam1997physical,msallam2000physical}.
One of the interesting characteristics of this instrument is the fact that the physical dimensions of the resonator vary while playing it.
Some synthesis techniques such as digital waveguides allow to approach the issue of dynamic resonator changes in a simple and computationally efficient way, and this feature has been extensively used in real-time sound synthesis \cite{cook2002real}.

However, when modelling the resonator of the instrument using finite-difference time-domain (FDTD) methods, or finite difference schemes (FDSs), the issue is not as trivial, and changes might compromise stability.
Previous implementations of brass instruments using FDSs focus on the trumpet 
 \cite{harrison2015environment}. To our knowledge, the simulation of a trombone with varying in realtime the shape of the resonator using FDSs has not been addressed yet.
We can cope with this limitation  by having a grid that 
 dynamically changes as shown in a companion paper \cite{Willemsen2021}.

Briefly described, we modify the grid configurations of the FDSs by adding and subtracting grid points based on parameters describing the system.
In this paper we propose a full simulation of a trombone, describing in details all its elements and with a specific focus on the dynamic grid simulation.

This paper is structured as follows: Section \ref{sec:continuous} presents the models for the tube and lip reed in continuous time. Section \ref{sec:discrete} briefly introduces FDTD methods and then discretises the aforementioned continuous equations. Section \ref{sec:dynamicGrid} presents the dynamic grid used to simulate the trombone slide and details on the implementation are provided in Section \ref{sec:implementation}. Section \ref{sec:resDisc} presents the results of the simulation, and finally, some conclusory remarks appeark in Section \ref{sec:conclusion}.