\section{Introduction}\label{sec:introduction}

The trombone is a musical instrument that presents distinct challenges from the perspective of physical modelling synthesis.
In particular, the excitation mechanism between the lips and the player has been extensively studied, and simulated mostly using a simple mass-spring damper system \cite{campbell2004brass}.
Because the majority of the bore is cylindrical, nonlinear effects can appear at high blowing pressures \cite{Hirschberg96}, leading to changes in timbre, or brassiness, and such effects have been investigated and simulated
\cite{campbell2004brass, msallam1997physical,msallam2000physical}.
However, the defining characteristic of the trombone is that the physical dimensions of the resonator vary during playing.
Synthesis techniques such as digital waveguides allow an approach to dynamic resonator changes in a simple and computationally efficient way, simply by varying the length of the corresponding delay line. This feature has been used in real-time sound synthesis \cite{cook2002real}, for simplified bore profiles suitable for modelling in terms of travelling waves.

However, when attempting more fine-grained modelling of the trombone resonator using finite-difference time-domain (FDTD) methods, the issue of the change in the tube length is not trivial. Previous implementations of brass instruments using these methods focus on the trumpet \cite{harrison2015environment} and various brass instruments (including the trombone bore) under static conditions \cite{Bilbao2013}. To our knowledge, the simulation of a trombone varying the shape of the resonator in real time using FDTD methods has not been approached.
We can tackle this problem by having a grid that dynamically changes while the simulation is running as presented in a companion paper \cite{Willemsen2021}. Briefly described, we modify the grid configurations of the FDTD method by adding and removing grid points based on parameters describing the system. 

In this paper we propose a full simulation of a trombone, describing all its elements in detail with a specific focus on the dynamic grid simulation. Section \ref{sec:continuous} presents the models for the tube and lip reed interaction in continuous time. Section \ref{sec:discrete} briefly introduces FDTD methods and the discretisation of the aforementioned continuous equations. Section \ref{sec:dynamicGrid} presents the dynamic grid used to simulate the trombone slide and details on the implementation are provided in Section \ref{sec:implementation}. Section \ref{sec:resDisc} presents simulation results, and some concluding remarks appear in Section \ref{sec:conclusion}.